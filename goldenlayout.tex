\documentclass[a4paper,10pt]{scrarticle}
\usepackage{fontspec}

\usepackage{polyglossia}
\setmainlanguage[variant=british]{english}
\setotherlanguage[babelshorthands=true]{german}

\defaultfontfeatures{Scale=MatchLowercase}
\setmainfont[Ligatures=TeX]{Linux Libertine O}
\setsansfont[Ligatures=TeX]{Linux Biolinum O}

\usepackage{xcolor}
\usepackage{booktabs}
\usepackage[fontsize=12pt]{goldenlayout}


\setlength{\parindent}{0pt}
\reversemarginpar

\title{\textnormal{The \textbf{goldenlayout} Package}}
\author{Alexander Bernardi}

\begin{document}
\maketitle

\section{Introduction}

\subsection{Typography}

The aim of the \textbf{goldenlayout} class is to fit the text within a given content width by using the golden ratio.
The golden ratio forms the basis for the design in terms of size, spacing and typography. 

\begin{itemize}
	\item a typographic scale based on the primary font size
	\item corresponding line heights for the typographic scale
	\item spacing units based on the line height associated with the primary font size
\end{itemize}

The golden ratio is defined as

\begin{equation}
\phi = \frac{1+\sqrt{5}}{2} = 1,6180339887
\end{equation}


\subsubsection{Fontsize}

For each primary font size f, we can apply the golden ratio to determine a typographic scale with different font sizes

\begin{equation}
f = f\phi^{\frac{n}{2}}
\end{equation}

\begin{table}
\begin{tabular}{ll}
{\Huge Huge} & {\Huge \getDocumentFontsize\,@\,\the\baselineskip} \\
{\huge huge} & {\huge \getDocumentFontsize\,@\,\the\baselineskip} \\
{\LARGE LARGE} & {\LARGE \getDocumentFontsize\,@\,\the\baselineskip} \\
{\Large Large} & {\Large \getDocumentFontsize\,@\,\the\baselineskip} \\
{\large large} & {\large \getDocumentFontsize\,@\,\the\baselineskip} \\
{\normalsize normal} & {\normalsize \getDocumentFontsize\,@\,\the\baselineskip} \\
{\small small} & {\small \getDocumentFontsize\,@\,\the\baselineskip} \\
{\footnotesize footnotesize} & {\footnotesize \getDocumentFontsize\,@\,\the\baselineskip} \\
{\scriptsize scriptsize} & {\scriptsize \getDocumentFontsize\,@\,\the\baselineskip} \\
{\tiny tiny} & {\tiny \getDocumentFontsize\,@\,\the\baselineskip} \\
\end{tabular}
\caption{Font size scale and corresponding line height}
\end{table}

\subsubsection{Line height}

For each font size f, a golden line height $h_{\phi}$ can be expressed as follows

\begin{equation}
h_{\phi} = f\phi
\end{equation}

Experience and research show, as the length of the text lines increases, the line height must also increase in order to maintain proportionality and thus readability.
As a consequence, the line height must be adapted to the context in which it is presented, i.e. the line height must be adapted to the width of the content.

\begin{equation}
h = f\left[(3-\phi)+(2\phi-3)\frac{w}{fx_{w}}+\frac{x-(\phi-1)}{\phi}\right]
\end{equation}

\section{Options}

The \textbf{goldenlayout} can be customised by optional global parameters.

\noindent\marginpar{fontsize} 

\noindent\marginpar{chars per line}

\noindent\marginpar{widthfactor}


\section{Commands}

\noindent\marginpar{milTime} The \textbf{milTime} command sets the time in 24 hour format hhmm without delimiter. The time zone is optional.

\end{document}